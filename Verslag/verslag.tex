\documentclass[titlepage, a4paper]{article}
\usepackage[utf8]{inputenc}
\usepackage{amsmath}
\usepackage{amssymb}
\usepackage{amsthm}
\usepackage{hyperref}
\usepackage{cleveref}
\usepackage{geometry}
\usepackage{enumerate}
\usepackage{mathtools}
\usepackage{tikz-cd}
\usepackage[backend = biber]{biblatex}
\addbibresource{references.bib}

\makeatletter
\def\input@path{{../Bewijzen/}}

\makeatother

\author{Lucas Levrouw \and Michaël Maex \and Tomas Reunbrouck \and Supervisor: Gabor Szabo}

\title{Kaplansky's Conjectures and Sofic Groups}

\date{\today}

\newcommand{\N}{\mathbb{N}}
\newcommand{\Z}{\mathbb{Z}}
\newcommand{\Q}{\mathbb{Q}}
\newcommand{\R}{\mathbb{R}}
\newcommand{\C}{\mathbb{C}}
\newcommand{\F}{\mathbb{F}}

\newcommand{\card}[1]{\left| #1 \right|}
\newcommand{\inclusion}[1]{\xhookrightarrow{#1}}
\newcommand{\surjection}[1]{\xtwoheadrightarrow{#1}}
\DeclareMathOperator{\sym}{Sym}
\DeclareMathOperator{\im}{im}



\newtheorem{theorem}{Theorem}
\newtheorem{definition}{Definition}
\newtheorem{lemma}{Lemma}
\newtheorem{conjecture}{Conjecture}
\newtheorem{corollary}{Corollary}
\theoremstyle{remark}
\newtheorem{example}{Example}
\newtheorem{remark}{Remark}

\usepackage{import}
\usepackage{xifthen}
\usepackage{pdfpages}
\usepackage{transparent}

\newcommand{\incfig}[1]{%
    \def\svgwidth{.5\columnwidth}
    \import{../Figuren/}{#1.pdf_tex}
}

\begin{document}
\pagenumbering{Alph}
    \maketitle
\pagenumbering{arabic}
\tableofcontents
\pagebreak

\section{Introduction}\label{sec:intro}
This thesis discusses the conjecture of Kaplansky which states that any group ring K[G] with K a field is directly finite. That is, the left-invertibility of elements implies their right-invertibility. For general groups and fields, this conjecture remains unanswered. It has been resolved however for specific cases, such as the case in which the group G is sofic (the definition of soficity follows in section 3). Some cases require the assumption of another conjecture. For example, when we talk about finite fields K we have a positive result for Kaplansky’s conjecture assuming that all countable groups are surjunctive (definition in section 5). This assumption is known as the conjecture of Gottschalk and will be discussed in detail.\\
\\
We shall mainly focus on the following two concepts: soficity and surjunctivity. First we shall discuss what it means for a group to be sofic. It is a very broad concept. In fact, there exist no known examples of non-sofic groups \cite{weiss_2000}. This will be illustrated by proving soficity for a few large classes of groups. Furthermore we shall discuss its connections to other important concepts and group properties such as amenability. Secondly, this paper will cover surjunctivity which is related to soficity but will be used to discuss the connection between the conjectures of Kaplansky and Gottschalk. Specifically, it is known that all sofic groups are surjunctive as well.

\section{Group rings}\label{sec:group_rings}

The setting of Kaplansky's direct finiteness conjecture is about so-called group rings. These are defined as follows


\begin{definition}\label{def:group_ring}
    Let $K$ be a field and $G$ be a group. The group ring $K[G]$ is the set of formal linear combinations
    \[
        \sum_{g \in G} \lambda_g g,
    \]
    where only finitely many $\lambda_g \neq 0$ with operations $+$ and $\cdot$ defined by
    \begin{align*}
        \sum_{g \in G} \lambda_g g + \sum_{g \in G} \mu_g g
        &= \sum_{g \in G} (\lambda_g+\mu_g) g \\
        \left(\sum_{g \in G} \lambda_g g \right) \cdot \left(\sum_{g \in G} \mu_g g \right)
        &= \sum_{g \in G} \left( \sum_{h \in G} \lambda_{h} \mu_{h^{-1}g} \right) g.
        %\sum_{g \in G} \left( \sum_{\substack{g_1, g_2 \in G \\ g_1 g_2 = g}} \lambda_{g_1} \mu_{g_2} \right) g.
    \end{align*}
\end{definition}


\begin{theorem}
    Let $K$ be a field and $G$ a group. Then $K[G]$ is a unital ring.
\end{theorem}


\begin{proof}
    We need to prove associativity of $+$, commutativity of $+$, existence of zero element, existence of additive inverses, associativity for $\cdot$, left distributivity, right distributivity and existence of a unit element.
    % further proof needed
    \begin{description}
    \item[associativity of $+$]
    \begin{align*}
    \left(\sum_{g \in G} \lambda_g g + \sum_{g \in G} \mu_g g\right)+ \sum_{g \in G} \nu_g g
    &= \sum_{g \in G} \left((\lambda_g g + \mu_g g)+\nu_g g\right) \\
    &= \sum_{g \in G} \left(\lambda_g g + (\mu_g g+\nu_g g)\right) \\
    &= \sum_{g \in G} \lambda_g g + \left(\sum_{g \in G} \mu_g g + \sum_{g \in G} \nu_g g \right)
    \end{align*}

    \item[commutativity of $+$]
    \begin{align*}
    \sum_{g \in G} \lambda_g g + \sum_{g \in G} \mu_g g
    &= \sum_{g \in G} (\lambda_g g + \mu_g g) \\
    &= \sum_{g \in G} (\mu_g g + \lambda_g g) \\
    &= \sum_{g \in G} \mu_g g + \sum_{g \in G} \lambda_g g 
    \end{align*}

    \item[existence of $0$]
    \begin{align*}
    \sum_{g \in G} \lambda_g g + \sum_{g \in G} 0 g
    &= \sum_{g \in G} (\lambda_g + 0)g \\
    &= \sum_{g \in G} \lambda_g g
    \end{align*}

    \item[existence of additive inverse]
    \begin{align*}
    \sum_{g \in G} \lambda_g g + \sum_{g \in G} (-\lambda_g g)
    &= \sum_{g \in G} (\lambda_g - \lambda_g) g \\
    &= 0
    \end{align*}

    \item[associativity of $\cdot$]
    \begin{align*}
    \left[\left(\sum_{g \in G} \lambda_g g \right) \cdot \left(\sum_{g \in G} \mu_g g \right)\right ] \cdot \sum_{g \in G} \nu_g g
    &= \left(\sum_{g \in G} \left( \sum_{h \in G} \lambda_{h} \mu_{h^{-1}g} \right) g \right) \cdot \sum_{g \in G} \nu_g g\\
    &= \sum_{g \in G} \sum_{x \in G} \left( \sum_{h \in G} \lambda_{h} \mu_{h^{-1}x} \right) \nu_{x^{-1}g} \,g \\
    &= \sum_{g \in G} \sum_{y \in G} \sum_{h \in G} \lambda_{h} \mu_{y} \nu_{y^{-1} h^{-1}g} \, g \\
    &= \sum_{g \in G} \sum_{h \in G} \lambda_h \left(\sum_{y \in G}\mu_{y} \nu_{y^{-1} (h^{-1}g)}\right)g \\
    &= \left(\sum_{g \in G} \lambda_g g \right) \cdot \left[ \left(\sum_{g \in G} \mu_g g \right) \cdot \sum_{g \in G} \nu_g g \right]
    \end{align*}

    \item[existence of 1]
    \begin{align*}
    \sum_{g \in G} \delta_{g,e} g \cdot \sum_{g \in G} \lambda_g g
    &= \sum_{g \in G} \left(\sum_{h \in H} \delta_{h,e} \lambda_{h^{-1}g} \right) \\
    &= \sum_{g \in G} \lambda_{e^{-1}g} g = \sum_{g \in G} \lambda_g.
    \end{align*}
    
        
    \end{description}
\end{proof}

\begin{remark}
	A group ring in general is not commutative. In fact $\mathbb{K}[G]$ is commutative if and only if $G$ is a commutative group. So we will have to be careful when dealing with these rings. 
\end{remark}
We can now state Kaplansky's conjecture.

\begin{conjecture}[Kaplansky]
    Let $K$ be a field and $G$ a group. Then $K[G]$ is directly finite i.e. for all $a, b \in K[G]$, the equation $ab=1$ implies $ba=1$.
\end{conjecture}

To put it shortly, Kaplansky's conjecture states that for all elements in the group ring, left-invertible implies right invertible. If $K$ is a finite field and $G$ is a finite group, it can quite easily be proven that $K[G]$ is directly finite.

\begin{theorem}
    Let $K$ be a field and $G$ be a finite group. Then $K[G]$ is directly finite.
\end{theorem}
\begin{proof}
    Let $a, b \in K[G]$ and suppose $ab=1$. Now consider the map
    \[
        \phi: K[G] \to K[G]: x \mapsto x a
    \]
    This map is injective. To see this, let $x, y \in K[G]$ such that $xa=ya$. Multiply on the right with $b$ and use the associativity to conclude $x=y$.
    But since $K[G]$ is a finite set, the map must also be surjective. Therefore there exists an $x \in K[G]$ such that $x a = 1$. Furthermore, $x = xab = b$. So we have $b=a$.
\end{proof}

Note how the argument hinges on the fact that the map $\phi$ is guarantied to be surjective by the finiteness of $G$. In what follows we will define several ``finiteness'' properties and try to generalise the proof above under weaker assumptions.

\section{Sofic groups}\label{sec:sofic_group}

\subsection{Definition and examples}

    A key notion of interest will be that of a sofic group. We call a group sofic if it can be ``well approximated by a finite permutation group". This is made precise in the following definition.

    \begin{definition}\label{def:Sofic}
        Let $G$ be a countable group. We say that $G$ is sofic if there exists a sequence $(d_i)_{i \in \N} \in \N ^{\N}$ with associated maps $\Phi_i : G \to \sym(d_i)$ such that $\forall\ g, h \in G$:
        \begin{enumerate}[(i)]
            %\item $\displaystyle \lim_{i\to \infty} \frac{1}{d_i} \card{\big\{k \in \{1, \dots, d_i\} \mid \sigma_{i,g} \circ \sigma_{i,h}(k) = \sigma_{i, g*h} (k) \big\}} = 1$
            %\item $\displaystyle \lim_{i\to \infty} \frac{1}{d_i}  \card{\big\{k \in \{1, \dots, d_i\} \mid \sigma_{i,g}(k) = k \big\}} = \delta_{e,g}$,
            \item $\displaystyle \lim_{i\to \infty} \frac{1}{d_i} \card{\big\{k \in \{1, \dots, d_i\} \mid (\Phi_i(g) \circ \Phi_i(h))(k) = \Phi_i(gh) (k) \big\}} = 1$
            \item $\displaystyle \lim_{i\to \infty} \frac{1}{d_i}  \card{\big\{k \in \{1, \dots, d_i\} \mid \Phi_i(g)(k) = k \big\}} = \delta_{e,g}$,
        \end{enumerate}
        where $\delta_{e,g}$ is $1$ if $g = e$ and $0$ otherwise.
    \end{definition}

    We will call such a family of maps $\left(\Phi_i : G \to \sym(d_i) \right)_{i \in \N}$ a \emph{sofic approximation} for $G$. It is unkown if every group is sofic.
    % Alternative definitions here (Capraro-Lupini definition, example 2.1.8)

    \begin{example}\label{ex:finite_group_sofic}
    Consider a finite group $G$. Look at the map $\phi: G \to \sym(G): g \mapsto \sigma_g$, where $\sigma_g$ is defined by $\sigma_g(h) = gh$. Clearly, $\phi$ is a homomorphism. Now we can choose $d_i = \card G$ and $\Phi_i = \phi$ for all $i \in \N$. Here we have identified $\sym(G)$ with $\sym(d_i)$. The first condition is satisfied since $\phi$ is a morphism. The second condition follows from the fact that $\sigma_g(h) = gh = h$ if and only if $g$ is the identity.
     Hence every finite group is sofic.
    \end{example}

    % Example: Z

	\begin{example}\label{ex:Z}
	Consider the group $\Z$. Take the sequence of maps $\Phi_{i}: G \to Sym(i): g \mapsto \Phi_{i}(g)$ where $\Phi_{i}(g)$ is defined as the map such that for any $k \in \{1,...,i\}$ : $ \Phi_{i}(g)(k) = (k+g)mod(i)$. The first condition of soficity is satisfied as for any $i \in N, g,h \in \Z, k \in \{1,..., i\}$ it can be said that 
	\begin{align*}
	(\Phi_{i}(g) \circ \Phi_{i}(h))(k) &= \Phi_{i}(g)((k+h)mod(i))\\
	&= (k+h+g)mod(i)\\
	&= (k+g+h)mod(i)\\
	&= \Phi_{i}(g+h)(k)
\end{align*}		
	Now for the second condition. It is clear that for any $i \in \N, k \in \{1,..., i\}$ : $\Phi_{i}(0)(k) = k$. Now take any $g \neq 0$ and consider $\Phi_{i}(g)$. For any $k \in \{1,...,i\}$ it is perfectly possible that $\Phi_{i}(g)(k) = k$. However this is not the case as soon as $i > |g|$. Then, namely, we have that $|(k+g)-k| < i$ which means that is is not possible that $(k+g)mod(i) = (k)mod(i)$. Thus we have that $(k+g)mod(i) \neq k$ as $(k)mod(i) = k$ for $k<i$ and the inequality is obvious for $k \geq i$.
	\end{example}
    % Example: Direct product of sofic groups is sofic. 

    \begin{example}\label{ex:direct_product_sofic}
        The direct product of two sofic groups is again sofic. Consider two sofic groups $G$ and $H$. Let $(\phi_i: G \to \sym(d_i))_{i \in \N}$ and $(\psi_i: G \to \sym(e_i))_{i \in \N}$ be sofic approximations for $G$ resp. $H$. Now consider $\Phi_i : G \times H \to \sym(d_i+e_i)$ for every $i \in \N$, where $\Phi_i(g, h) \in \sym(d_i+e_i)$ is defined by
        \[
            \Phi_i(g,h)(k) = \begin{cases} \phi_i(g)(k) & \text{if } 1 \leq k \leq d_i \\
            \psi_i(h)(k)+d_i & \text{if } d_i + 1 \leq k \leq e_i
            \end{cases}.
        \]
        It is not hard to show that $(\Phi_i)_{i \in \N}$ is a sofic approximation for $G \times H$. % Write out proof
        Hence finite direct products conserve the soficity property.
    \end{example}


    Often, it will be more convenient to check a property that implies soficity than checking it directly as we have done in the examples above. We will discuss two such properties: amenability and residually finiteness.

    \subsection{Amenability}



    \begin{definition}\cite{noauthor_folner_2019} \label{def:folner}
        Let $G$ be a countable group. We say that $G$ satisfies the Følner criterion if there exists a sequence of finite subsets (not necessarily subgroups) $F_1, F_2, \dots \subset G$ such that 
        \begin{enumerate}
            \item for every $g \in G$ there exists an $N \in \N$ such that for all $n > N: x \in F_n$
            \item for all $g \in G$ 
            \[\lim_{i\to \infty} \frac{\card{gF_i \triangle F_i}}{\card{F_i}} = 0, \]
            where $\triangle$ denotes the symmetric difference.
        \end{enumerate}
    \end{definition}
\begin{figure}[ht]
    \centering
    \incfig{folner}
    \caption{An illustration of the Folner criterion. The limit means that the gray part will be come very small compared to  $F_n$ itself as $n$ gets large.}
    \label{fig:riemmans-theorem}
\end{figure}
% example: finite group
If a countable group statisfies the Følner condition, we will also call the group \emph{amenable}.\footnote{One can also define amenability also for uncountable groups. In that case however, it is defined differently and the Følner criterion is not equivalent the amenability.}

\begin{example}\label{ex:finite_group_folner}
    Every finite group trivially satisfies the Følner condition. Indeed, we can take $F_i = G$ for every $i \in \N$. Then the first condition is automatically statisfied and we also have $gG \triangle G = G \triangle G = \emptyset$. Hence $\card{gF_i\triangle F_i} = 0$, for all $i \in \N$.
\end{example}

\begin{example}\label{ex:integers_folner}
    The group of integers $\Z$ is also amenable. Let $F_i = \{-i, -i+1, \dots, i-1, i\}$ for every $i \in \N$. The first condition is satisfied. For the second condition, let $k$ be an arbitrary integer. Then
    \[
        \frac{\card{(k + F_i) \triangle F_i}}{\card{F_i}}
        = \frac{\card{\{k-i, k-i+1, \dots, k+i-1, k+i\}\triangle \{-i, -i+1, \dots, i-1, i\}}}{2i+1}
        \leq \frac{2k}{2i+1},
    \]
    and this converges to zero as $i \to \infty$.
\end{example}

% example: finitely generated groups are not amenable

To prove that each group that satisfies the Følner criterion is sofic, we first need the following set-theoretic lemma.

\begin{lemma}\label{lem:finite_bijections} 
        Let $A$ and $B$ be finite sets such that $|A| = |B|$. Given $A' \subset A, B' \subset B$ and a bijection $f': A' \to B'$ there exists a bijection $f: A \to B$ such that $f|_{A'} = f'$. 
    \end{lemma}
    \begin{proof}
        Since $\card A = \card B$ and $\card A' = \card B'$,
        \[
        \card{A \setminus A'} = \card A - \card A' = \card B - \card B' = \card{B \setminus B'}.
        \]
        Therefore there exists a bijection $g: A \setminus A' \to B \setminus B'$. Define $f: A  \to B$ by $f(x) = f'(x)$ for all $x \in A'$ and $f(x)=g(x)$ for all $x \in A \setminus A'$. Then $f$ is a bijection and $f|_{A'} = f'$.
    \end{proof}

We now state the proof.

 	\begin{theorem}\label{thm:folner_sofic}
        Every group $G$ that satisfies the Følner criterion is sofic.
    \end{theorem}
    \begin{proof}
        Let $F_1, F_2, F_3, \dots $ be a Følner sequence of $G$.
%Notice that $G = \bigcup_{i \in \N} F_i$ by the first property. So $G$ is countable. 
Choose for every $i \in \N: d_i = |F_i|$. We will implicitly identify $\sym(F_i)$ with $\sym(d_i)$. Now choose maps $\Phi_i: F_i \to \sym(F_i)$ such that $\Phi_i(g)(h) = gh$ whenever $gh \in F_i$, which is possible due to \cref{lem:finite_bijections}. 

We first prove the second condition.
Fix $g, h \in G$. By the Følner criterion we know that 
\begin{align*}
    \lim_{i\to \infty} \frac{\card{g^{-1}F_i \triangle F_i}}{\card{F_i}} &= 0\\
    &= \lim_{i\to \infty} \frac{\card{g^{-1}F_i \setminus F_i} + \card{F_i \setminus g^{-1}F_i}}{d_i}\\
    &= \lim_{i\to \infty} \frac{2\card{ F_i \setminus g^{-1}F_i }}{d_i}\\
    &= \lim_{i\to \infty} \frac{2\card{\left\{ x \in F_i \mid gx \notin F_i \right\}}}{d_i}. 
\end{align*}
This implies that \begin{equation*}
    \lim_{i\to \infty} \frac{\card{\left\{ x \in F_i \mid gx \in F_i \right\}}}{d_i} = 1,
\end{equation*}
or equivalently 
\begin{equation*}%\label{eq:proof_folner1}
    \lim_{i\to \infty} \frac{\card{\left\{ x \in F_i \mid gx \notin F_i \right\}}}{d_i} = 0
\end{equation*}
As $\Phi_i(g)(x) = gx$ whenever $gx \in F_i$, we can see as well that 
$$\lim_{i\to \infty} \frac{\left|\left\{ x \in F_i \mid \Phi(g)(x)\ne gx \right\}\right|}{d_i} = 0.$$
Which proves the second condition for soficity. 

To prove the first condition, we can apply \cref{lem:folner_finite_subset}, which gives that \[
	\lim_{i \to \infty} \frac{\card{\{g,h, gh\} F_i \triangle F_i}}{d_i} = 0
.\]
Which means that 
$$\lim_{i\to \infty} \frac{\left|\left\{ x \in F_i \mid gx,hx,ghx \in F_i \right\}\right|}{d_i}= 1.$$
As $\Phi_i(g)(x) = gx$ whenever $gx \in F_i$, we can see as well that
$$\lim_{i\to \infty} \frac{\left|\left\{ x \in F_i \mid (\Phi(g) \circ\Phi(h))(x) = \Phi(gh)(x) \in F_i \right\}\right|}{d_i}= 1.$$
This proves the first condition for soficity and concludes the proof. 

 	\end{proof}

    \begin{theorem}\label{thm:countable_abelian_folner}
        Every countable abelian group satisfies the Følner criterion. 
    \end{theorem}
    \begin{proof}
        The group $G$ is countable so we can write $G = \{e, g_1, g_2, g_3, \dots\}$.
Consider the sequence of sets $$F_n = \left\{\prod_{i = 1}^n g_i^{a_i} \mid \card{a_i} \le n\right\}.$$ 
It is clear that this set adheres to the first condition of the folner criterion. 
Fix $g_j \in G$. Take any $n > j$.
Let $$H = \left\{\prod_{i = 1, i \ne j}^n g_i^{a_i} \mid \card{a_i} \le n\right\}.$$
Notice that $\bigcup_{i = -n}^{n} g^{i}H = F_n$.
Define the finite sequence of sets $(B_m)_m$ for $m \in \{-n, \dots, n+1\}$ such that  
$B_m = g^mH \setminus \bigcup_{i=-n}^{m-1}g^iH$.
It should be clear that the elements of $(B_n)$ are pairwise disjoint and that  \[
    \bigcup_{i = -n}^n B_n = F_n 
.\] 
It follows that $\sum_{i=-m}^{m} \card{B_i} = \card{F_n}$. We know that 
\begin{align*}
    \card{gF_n \triangle F_n} &= 2 \card{gF_n \setminus F_n} \\
    &= 2 \card{\bigcup_{i = -n}^n g^{i+1}H \setminus \bigcup_{i = -n}^n g^{i}H}\\
    &=  2 \card{g^{n+1}H \setminus \bigcup_{i = -n}^n g^{i}H}\\
    &= 2 \card{B_{n+1}}
\end{align*}
So to show that the $F_n$ adheres to the second condition of the folner criterion, it is sufficient to show that $\frac{\card{B_{n+1}}}{\card{F_n}} \to 0$ as $n \to \infty$.

We claim that $(\card{B_m})$ is a non-increasing sequence. The following calculation shows this.
\begin{align*}
    \card{B_m} &= \card{g^{m}H \setminus \bigcup_{i=-n}^{m-1}g^iH}\\
    &= \card{g^{m-1}H \setminus \bigcup_{i=-n}^{m-1}g^{i-1}H}\\
    &= \card{g^{m-1}H \setminus \bigcup_{i=-n-1}^{m-2}g^{i}H}\\
    &\le \card{g^{m-1}H \setminus \bigcup_{i=-n}^{m-2}g^{i-1}H} = \card{B_{m-1}}
\end{align*}
So $\card{F_n} = \sum_{i=-n}^{n} \card{B_i} \le (2n+1) \card{B_{n+1}}$, which means that $$\frac{\card{B_{n+1}}}{\card{F_n}} \le \frac{1}{2n+1}.$$
So we find that $\frac{\card{B_{n+1}}}{\card{F_n}} \to 0$ which concludes the proof. 
    \end{proof}
    An immediate corollary is:
    \begin{corollary}
    	Every countable abelian group is sofic.	
    \end{corollary}
    % \begin{lemma}
    %     \label{lem:amenable_short_exact_sequence}
    %     Let $A, B, C$ be groups with morphisms $f:A\to B, g:B\to C$ such that
    %     \[\begin{tikzcd}
    %         A \arrow[r, "f", hook] & B \arrow[r, "g", two heads] & C
    %     \end{tikzcd}\] is a short exact sequence. Then $A$ and $B$ are amenable if and only if $C$ is amenable.
    % \end{lemma}
	The following lemma is a known fact about about ameanable groups, that we will use without proof.
    \begin{lemma}
        \label{lem:amenable_short_exact_sequence}
        Let $A, B, C$ be groups and let $f: A\to B$ be an injective morphism and $g:B\to C$ a surjective morphism such that 
        \[\begin{tikzcd}
            A \arrow[r, "f", hook] & B \arrow[r, "g", two heads] & C
        \end{tikzcd}\]
        is an exact sequence i.e. $\im f = \ker g$. Then $A$ and $B$ are amenable if and only if $C$ is amenable.
    \end{lemma}

    \begin{theorem}
        Every countable solvable group is amenable. 
    \end{theorem}
    \begin{proof}
        Let $G$ be a countable solvable group. Then there exists a finite sequence of subgroups such that 
$$\{e\} = G_0 \triangleleft G_1 \triangleleft G_2 \triangleleft \dots \triangleleft G_n = G,$$ and for every $i, 0\le i<n$, the quotient $\frac{G_{i+1}}{G_i}$ is commutative. 
It is clear that $G_0$ is amenable. We will now use finite induction to show that $G$ is amenable as well. 
Take any $i, 0\le i<n$ and assume that $G_i$ is amenable. Then 
\[\begin{tikzcd}
    G_i \arrow[r, "\id", hook] & G_{i+1} \arrow[r, "\pi", two heads] & \frac{G_{i+1}}{G_i}
    \end{tikzcd}\]
is an exact sequence. Because $\frac{G_{i+1}}{G_i}$ is commutative we know by \cref{thm:countable_abelian_folner} that it is amenable. We assumed that $G_i$ is amenable as well. By \cref{lem:amenable_short_exact_sequence} we find that $G_{i+1}$ is amenable as well. This finishes the induction and thus we can conclude that $G_n = G$ is amenable.

    \end{proof}

\subsection{Residually finiteness}

Another property that implies soficity is the following.

\begin{definition}\cite{noauthor_residually_2018} \label{def:res_fin}
       We say that a group $G$ is residually finite if it adheres to one of the following equivalent properties.
        \begin{enumerate}
            \item For every $g \in G\backslash\{e\}$ there exists a finite group $H$ and a homomorphisms $f:G \to H$ such that $f(g) \ne e$.
            \item For every $g \in G\backslash\{e\}$ there exists a normal subgroup $H$ such that $g \notin H$.
            \item The intersection of all subgroups of finite index is $\{e\}$.
            \item The intersection of all normal subgroups is $\{e\}$.
            \item $G$ can be embedded in a direct product of finite groups.
        \end{enumerate}
    \end{definition}

% Proof that these are equivalent? Which ones do we actually use?

% example: finite groups

% example: Z

% example: finitely generated groups

    \begin{theorem} \label{thm:res_fin_sofic}
        Every countable residually finite group is sofic.
    \end{theorem}
    \begin{proof}
        Let $G$ be a countable residually finite group. As $G$ is countable we can order the elements non-identity elements $g_1, g_2, g_3, \dots$ .
By the first property of residually finiteness, for every $i=1,2,3, \dots$ there exist a group $G_i$ and a homomorphism $f_i: G \to G_i$ such that $f_i(g_i) \ne 1_G$. Consider the direct products $H_n = \prod_{i = 1}^n G_i$ with the induced homomorphisms $h_i: G \to H_i: g \mapsto (f_1(g), f_2(g), \dots, f_i(g))$. Now we will look at the image of $G$ under these morphisms $\Gamma_i = h_i(G)$.


Notice that $\Gamma_i$ is always finite. We will implicitly identify $\sym(\Gamma_i)$ with $\sym(\card{\Gamma_i})$.
For every $i \in \N$, define $d_i = \card{\Gamma_i}$ and $\Phi_i: G \to \sym(\Gamma_i): g \mapsto (a \mapsto ga)$. We claim that this is a sofic approximation sequence for $G$.

For every $k \in \Gamma_i, g, h \in G$ we can see that $(\Phi_i(g) \circ \Phi_i(h))(k) = ghk = \Phi_i(gh)$. So
\[
\{k \in \Gamma_i \mid (\Phi_i(g) \circ \Phi_i(h))(k) = \Phi_i(gh)\} = \Gamma_i.
\]
Then we can see that for all $g,h \in G$:
$$\lim_{i\to \infty} \frac{1}{d_i} \left|\left\{k \in \Gamma_i| (\Phi_i(g) \circ \Phi_i(h))(k) = \Phi_i(gh)\right\}\right| = 1.$$

For every $g \in G, g\ne 1$ there is an $N \in \N$ such that for every $i > N$, $h_i(G) \ne 1_G$. So it is clear that 
$$\lim_{i \to \infty} \frac{1}{d_i} \card{\left\{k \in \Gamma_i \mid \Phi_i(g) = gk \ne k\right\}} = 1.$$
For $g = 1_G$ we find that 
$$\lim_{i \to \infty} \frac{1}{d_i} \card{\left\{k \in \Gamma_i \mid \Phi_i(g) = k \ne k\right\}} = 0,$$
which proves the second condition. 

    \end{proof}

    \subsection{Examples of Sofic Groups}
    \begin{description}
        \item[Finite groups] Every finite group, $G$ satisfies the Følner criterion, as the taking $(G)_n$ as sequence satisfies the properties in \cref{def:folner}.
        \item[Cyclic groups] All finite cyclic groups are taken care of by the previous example. We only need to check $\Z,+$. This group is residually finite as for every $n \in \Z, n \ne 0$ we can consider the natural morphism $\Z \to \Z /(|n|+1)\Z: x \mapsto x \mod |n|+1$. 
        \item[The finite direct product of Sofic groups] Let $G, H$ be sofic groups with sequence $d_i, e_i$ and maps $\sigma_i, \phi_i$ such that the properties of \cref{def:Sofic} hold. It can be (easily) verified that the sequence $(d_i + e_i)_i$ and maps $\psi_i: G \times H \to \sym(d_i + e_i): x (x \le d_i) \mapsto \sigma_i(x), x (x > d_i) \mapsto \phi_i(x -d_i) + d_i$ satisfy \cref{def:Sofic}.
        \item[Finitely generated free groups] They are residually finite.
    \end{description}
    \paragraph{conjectures}
    \begin{itemize}
        \item subgroups of sofic groups
        \item free product of sofic groups
        \item direct product/sum of infinite groups
    \end{itemize}

 

    \section{Surjunctivity}\label{sec:surjunctivity}

    First we define a basic concept in group theory.

    \begin{definition} % van wikipedia gehaald
        Let $G$ be a group and $X$ be a set. Then $\alpha: G \times X \to X: (g,x) \mapsto \alpha_g(x)$ is called an action of $G$ on $X$ if
        \begin{enumerate}[(i)]
            \item $\forall x \in X: \alpha_e(x) = x$
            \item $\forall g,h \in G, x\in X: \alpha_{gh}(x) = \alpha_g(\alpha_h(x))$.
        \end{enumerate}
    \end{definition}

    In this section we will encounter the set of all maps from $G$ to a finite set $A$. This set will be denoted by $A^G$. We will consider this set as a topological space, the topology on $A^G$ being the product topology, where $A$ carries the discrete topology.

    \begin{definition}
        Let $G$ be a countable group and $A$ a finite set. Then the right shift $\sigma$ of $G$ on $A$ is defined by
        \[
        \sigma: G \times A^G \to A^G: \left(g, (x_{h})_{h \in G} \right) \mapsto \sigma_g(\omega),
        \]
        where $\sigma_g$ is defined by $\sigma_g\left( (x_{h})_{h \in G} \right) = \left( x_{hg} \right)_{h \in G}$. % Misschien is de ``rij-notatie'' hier niet aangewezen
    \end{definition}

    One can see that $\sigma$ is a $G$-action on $A^G$.
    \begin{definition}
        % Let $G$ be a countable group and $A$ a finite set. 
        % Let $\sigma: G\to \hom{A^{G}}:g \mapsto \left( (x_{h})_{h \in G} \mapsto \left( x_{gh} \right)_{h \in G}  \right)$.
        % The group $G$ is called surjunctive if every injective and continuous $\phi: A^{G}\to A^{G}$ that commutes with $\sigma$ is surjective.
        Let $G$ be a countable group and $A$ a finite set. The group $G$ is called surjunctive if every injective and continuous $\phi: A^{G}\to A^{G}$ that commutes with $\sigma_g$ for all $g \in G$ is also surjective.
    \end{definition}

    \begin{conjecture}[Gottschalk] \label{conj:gottschalk}
        Every countable group is surjunctive.
    \end{conjecture}
    % \begin{conjecture}[Gottschalk] \label{conj:gottschalk}
    %   Let $G$ be a countable group and $A$ a finite set. 
    %   Let $\sigma: G\to \hom{A^{G}}:g \mapsto \left( (x_{h})_{h \in G} \mapsto \left( x_{gh} \right)_{h \in G}  \right)$.
    %   Then every injective and continuous $\phi: A^{G}\to A^{G}$ that commutes with $\sigma$ is surjective.
    % \end{conjecture}

    It can be proven that every sofic group is surjunctive. However, the proof is not easy. For that reason we will prove it first for the special case of residually finite groups. We will need the following definition

    \begin{definition}
        Let $G$ be a countable group and $A$ a finite set. Let $H$ be a subgroup of finite index. Then the set of $H$-periodic points is the following subset of $A^G$.
        \[
        P_H = \{ \omega \in A^G  \mid \forall h \in H: \sigma_h\omega = \omega\}
        \]
    \end{definition}

    Since $H$ is of finite index, this set is finite. % more explanation needed
    A key lemma we will further need in our proof of theorem \ref{thm:res_fin_surjunctive} is the following

    \begin{lemma} \label{lem:h-periodic_points}
    The $H$-periodic points are dense in $A^G$, where $A$ is finite set and $G$ is a residually finite countable group.
    \end{lemma}
    \begin{proof}
        Let $f:G \to A \in A^G$. We will prove that there exists a sequence of periodic points that converges to $f$. As $G$ is countable we can lists its elements $G = \{1, g_1, g_2, g_3,\dots\}$. 
        Let $F_n = \{1, g_1, g_2, \dots, g_n\}$. 
        As $G$ is residually finite, for every $g \in F_n^{-1} \cdot F_n,\ g \ne 1$, we can find a map $\pi_g: G \to H_g$, where $H_g$ is a finite group and $\pi_g(g) \ne 1$.  Consider the map \[\pi = \prod_{ \substack{g\in  F_n^{-1} \cdot F_n\\ g\ne 0}} \pi_g
        .\]
        Observe that $\im(\pi)$ is finite, thus $\ker\pi$ is a subgroup of finite index of $G$. 
        For $g_i, g_j \in F, i \ne j$ is is clear that $\pi(g_i) \ne \pi(g_j)$ as $\pi(g_j^{-1} g_i) \ne 1$. 
        This means that there exists a function $\alpha: \im(\pi) \to A$ such that for all $g_i \in F_n: (\alpha \circ \pi)(g_i) = f(g_i)$. 
        Notice that $\alpha \circ \pi \in A^G$. 
        Let $h \in \ker\pi$ and consider the right $h$ shift, $\sigma_h$. 
        Take any $g \in G$. 
        Then 
        \[\sigma_h(\alpha\circ\pi)(g) = (\alpha \circ \pi)(hg) = \alpha(\pi(h)\pi(g)) = \alpha(\pi(g)) = (\alpha\circ\pi)(g).\] Thus $\sigma_g(\alpha \circ \pi) = (\alpha\circ\pi)$.
        So $\alpha\circ\pi$ is a $\ker\pi$ periodic point. 

        For every $F_n =  \{1, g_1, g_2, \dots, g_n\}$ we can construct such a function $\pi_n$ which is an $H$ periodic point and that agrees with $f$ on $F_n$. It is clear that $(\pi_n)_{n\in\N}$ converges to $f$ pointwise as a function. This is sufficient to say that $(\pi_n)_n$ converges to $f$ in the product topology.
    \end{proof}

    Now we can quite easily prove the following result. We follow the proof of Weiss \cite{weiss_2000} here.
    \begin{theorem} \label{thm:res_fin_surjunctive}
    Every residually finite group is surjunctive.
    \end{theorem}
    \begin{proof}
    Let $G$ be a residually finite group and $A$ a finite set. Let $\phi: A^G \to A^G$ be a continuous map which commutes with the right shift $\sigma$. Assume further that $\phi$ is injective. We need to prove that $\phi$ is also surjective. Since $\phi$ commutes with the shift, it follows readily that it maps $H$-periodic points to $H$-periodic points, for any subgroup $H$ of finite index.
Because $P_H$ is finite (for $H$ of finite index), and $\phi$ is injective, the restriction of $\phi$ to $P_H$ is surjective, thus $P_H \subset \phi(A^G)$. Now since $A^G$ is compact, continuous maps from $A^G$ to itself map closed sets to closed sets. Therefore
\[
\phi(A^G) = \overline{\phi(A^G)} \supset
\overline{\bigcup_{\substack{\text{subgroup } H \\ [G : H] < \infty}}\phi( P_H)}
= \overline{\bigcup_{\substack{\text{subgroup } H \\ [G : H] < \infty}} P_H} = A^G,
\]
where we have used lemma \ref{lem:h-periodic_points}. Thus we conclude $\phi$ is surjective.

    \end{proof}

    % Example: Apply this proof to Z


	\section{Kaplansky's direct finiteness condition for finite fields}
	Let us recall the statement of Kaplansky's direct finiteness condition.
	\begin{definition}
		Let $K$ be a field and $G$ be a group. The group ring $K[G]$ is called directly finite if every  $a \in K[G]$, that is left invertible, is right invertible as well. 
	\end{definition}
	Kaplansky conjectured that all group rings are directly finite and Weiss showed that this is true for all sofic groups \cite{weiss_2000}. 
	In the case where $R$ is a finite field, Weiss did this by first proving the the surjunctivity of sofic groups.
	He finished the argument by proving that the group rings of finite fields and surjunctive groups are directly finite.  

	In this section we will give the details of a proof of that last step. 
	I.e we'll prove the following theorem. 
	\begin{theorem}\label{thm:gottschalk_kaplansky}
		Let $\F$ be a finite field and $G$ be a surjunctive group. Then $\F[G]$ is directly finite. 
	\end{theorem}
	It looks like we will need to use the surjuctivity on $\F^{G}$. Hence we need some statement that relates $\F^{G}$ to $\F[G]$.
	\begin{lemma}
		For a group $G$ and field $\mathbb K$ we can consider $\mathbb{K}^{G}$ as a left $\mathbb{K}[G]$-module, with pointwise addition and multiplication defined as
		\begin{align*}
			\cdot : (\mathbb{K}[G] , \mathbb{K}^{G}) &\longrightarrow \mathbb{K}^{G} \\
		\left( \sum_{g \in G} a_g g,  (x_g)_{g \in G}\right)  &\longmapsto \left( \sum_{h \in G} a_h x_{h^{-1}g} \right)_{g \in G}
	.\end{align*} 
	\end{lemma}
	\begin{proof}
		We check the module axioms one by one.
		In the following calculations $r = \sum_{g \in G}^{} r_g g,\ s = \sum_{g \in G}^{} s_g g$ are arbitrary elements of $\mathbb{K}[G]$ and $x = (x_g)_{g \in G}, y = (y_g)_{g \in G}$ are arbitrary elements of $\mathbb{K}^{G}$.
		\begin{description}
			\item[left distributivity] 
				\begin{align*}
					r\cdot (x+y) &= \left( \sum_{h \in G} r_h (x_{h^{-1} g} + y_{h^{-1} g}) \right)_{g \in G} \\
						     &= \left( \sum_{h \in G} r_h x_{h^{-1} g} \right)_{g \in G}  + \left( \sum_{h \in G} r_h y_{h^{-1}g} \right)_{g \in G}  \\
						     &= r\cdot x + r\cdot y 
				.\end{align*}
			\item[right distributivity]
				\begin{align*}
					(r + s)\cdot x &= \left( \sum_{h \in G} (r_h + s_h) x_{h^{-1} g} \right)_{g \in G}  \\
						       &= \left( \sum_{h \in G} r_h x_{h^{-1}g} \right)_{g \in G} + \left( \sum_{h \in G} s_{h} x_{h^{-1}g}  \right)_{g \in G} \\ 
						       &= r\cdot x + s \cdot y 
				.\end{align*}
		
			\item[associativity]
				 \begin{align*}
					 (rs)\cdot x &= \left( \sum_{g_1 \in G} \sum_{g_2 \in G} r_{g_1} s_{g_2} g_1 g_2 \right) \cdot (x_{g})_{g \in G} \\
						     &= \left( \sum_{h \in G} \sum_{k \in G} r_{k}s_{k^{-1}h} h\right) \cdot  \left( x_{g} \right)_{g \in G} \\
						     &= \left( \sum_{h \in G} \sum_{k \in G} r_k s_{k^{-1}h} x_{h^{-1}g} \right)_{g \in G}  \\
						     &= \left( \sum_{h \in G} \sum_{k \in G} r_k s_{k^{-1}h} x_{(k^{-1} h)^{-1}(k^{-1}g)} \right)_{g \in G}  \\
						     &= \left(\sum_{h \in G} r_h h\right) \cdot  \left( \sum_{k \in G} s_{k} x_{k^{-1}g} \right)_{g \in G} \\
						     &= r(s\cdot x) 
				.\end{align*}
			\item[unity]
				\begin{align*}
					1\cdot x &= (x_{1^{-1} g})_{g \in G} \\
						 &= x\\
				.\end{align*}
		\end{description}
	\end{proof}
	We're now ready to give a proof of \cref{thm:gottschalk_kaplansky}.
	\begin{proof}
		Let $\F$ be a finite field and $G$ be a surjunctive group. 
		Let $a = \sum_{g \in G} a_g g \in \F[G]$ be a left invertible element with left inverse $b$, i.e.  $b a = 1$. 
		Consider the map $\phi: \F^{G} \to \F^{G}: x \mapsto a\cdot x$. Notice that this map 
		\begin{itemize}
			\item is injective. Suppose that $\phi(x) = \phi(y)$. Then $a\cdot x = a\cdot y$. So $ba\cdot x  = x$ and $ba \cdot y = 1\cdot y = y$. Hence $x = y$.
			\item commutes with the bernoulli shift  $\sigma_k$. 
				 \[
					 \sigma_k \phi(x) 
					 = \sigma_k\left( \sum_{h \in G} a_h x_{h^{-1}g} \right)_{g \in G} 
					 = \left( \sum_{h \in G} a_h x_{h^{-1}gk} \right)_{g \in G} 
					 = \left( \sum_{h \in G}a_h h \right) \cdot (x_{gk})_{g \in G} = \phi(\sigma_k(x)) 
				\]
			\item is continuous. 
			One can easily check that the set $\left\{ U_{h, t} \right\}$, where \[
			U_{h,t}:= \prod_{g \in G} \begin{cases}
				\F & h \ne g\\
				\{t\}& h = g
			\end{cases}
			.\] 
			is a subbasis of $\F^{G}$. 
			Hence, to prove that $\phi$ is continuous it is sufficient to check that $\phi^{-1}(U_{h, t})$ is open for every $h \in G, t\in \F$.
			Notice that an  $(x_g)_{g \in G} \in \F^{G}$ is in $\phi^{-1}(U_{h,t})$ if and only if $\sum_{k \in G} a_k x_{k^{-1}h} = t$. 
			Remember that $a_k$ is nonzero only for a finite number subset $X$,  $k\in X \subset G$. 	
			So $\phi^{-1}(U_{h, t})$ is closed under addition with an element of  \[
			\prod_{g \in G} \begin{cases}
				\F & hg^{-1}  \not\in X\\
				\{0\} & hg^{-1} \in X 
			\end{cases}
			.\] 
			Let  $(x_g) \in \phi^{-1}(U_{g,t})$. We see that  \[
				(x_g) \in \prod_{g \in G} \begin{cases}
					\F & hg^{-1} \not\in X\\
					\{x_g\} & hg^{-1} \in X
				\end{cases}
				\subset \phi^{-1}(U_{h, t})
			.\] 
			So $\phi^{-1}(U_{h,t})$ is open. Hence $\phi$ is continuous.

				
		\end{itemize}
		By the surjunctivity of $G$ we find that $\phi$ is surjective as well. In particular this means that there exists a $y = (y_g)_{g \in G}$ such that $\phi(y) = (\delta_{g, 1})_{g \in G}$, where $\delta$ is the kronecker delta.
		By multiplying with $b$ on the left we find that \[
			 y = ba\cdot y = b\cdot (\delta_{g, 1})_{g \in G} = \left( \sum_{h \in G} b_{h} \delta_{h^{-1}g, 1} \right)_{g \in G} = \left( \sum_{h \in G} b_{h} \delta_{h, g} \right)_{g \in G}
		 = \left( b_{g} \right) _{g \in G}.\] 
		This means that \[
		a\cdot y = \left( \sum_{h \in G} a_h b_{h^{-1}g} \right)_{g \in G} = \left( \delta_{g, 1} \right) _{g \in G} 
	.\] 
	In other words
	\[
	\sum_{h \in G} a_h b_{h^{-1}g} = \delta_{g, 1}
	.\]
	But this means exactly that $ab = 1$. So  $a$ is right invertible as well. 
	\end{proof}	
	\printbibliography
\end{document}
