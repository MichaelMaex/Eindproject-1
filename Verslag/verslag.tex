\documentclass[a4paper]{report}
\usepackage[utf8]{inputenc}
\usepackage{amsmath}
\usepackage{amssymb}
\usepackage{amsthm}
\usepackage{hyperref}
\usepackage{cleveref}
\usepackage{geometry}
\usepackage{enumerate}
\usepackage[backend = biber]{biblatex}
\addbibresource{references.bib}

\makeatletter
\def\input@path{{../Bewijzen/}}
%or: \def\input@path{{/path/to/folder/}{/path/to/other/folder/}}
\makeatother

\author{Lucas Levrouw \and Michaël Maex \and Thomas Reunbrouck}

\title{Kaplansky's Conjectures and Sofic Groups}

\date{\today}

\newcommand{\N}{\mathbb{N}}
\newcommand{\Z}{\mathbb{Z}}
\newcommand{\Q}{\mathbb{Q}}
\newcommand{\R}{\mathbb{R}}
\newcommand{\C}{\mathbb{C}}

\newcommand{\card}[1]{\left| #1 \right|}
\DeclareMathOperator{\sym}{Sym}

\newtheorem{theorem}{Theorem}
\newtheorem{definition}{Definition}
\newtheorem{lemma}{Lemma}

\newtheorem{example}{Example}


\begin{document}
    \maketitle
    \chapter{Sofic groups}

\section{Definition and examples}

    \begin{definition}\label{def:Sophic}
        Let $\Gamma$ be a countable group. We say that $\Gamma$ is sofic if there exist $d_i \in \N$ and mappings $\sigma_i : \Gamma \to \sym(d_i)$ such that $\forall\ g, h \in \Gamma:$
        \begin{enumerate}[(i)]
            \item $\displaystyle \lim_{i\to \infty} \frac{1}{d_i} \card{\big\{k \in \{1, \dots, d_i\} \mid \sigma_{i,g} \circ \sigma_{i,h}(k) = \sigma_{i, g*h} (k) \big\}} = 1$
            \item $\displaystyle \lim_{i\to \infty} \frac{1}{d_i}  \card{\big\{k \in \{1, \dots, d_i\} \mid \sigma_{i,g}(k) = k \big\}} = \delta_{e,g},$
        \end{enumerate}
        where $\delta_{e,g}$ is 1 if $g = e$ and 0 otherwise.
    \end{definition}

    % Alternative definitions here (Capraro-Lupini definition, example 2.1.8)

    % Example: Z

    % Example: Direct product of sofic groups is sofic. 

    \section{Properties that imply soficity}
    \begin{definition}\cite{noauthor_folner_2019} \label{def:folner}
        Let $G$ be a group. We say that $G$ satisfies the Følner criterion if there exists a sequence of finite subsets (not necessarily subgroups) $F_1, F_2, \dots \subset G$ such that 
        \begin{enumerate}
            \item for every $g \in G$ there exists an $N \in \N$ such that for all $n > N: x \in F_n$
            \item for all $g \in G$ 
            $$\lim_{n\to \infty} \frac{\card{gF_i \triangle F_i}}{\card{F_i}} = 0, $$
            where $\triangle$ denotes the symmetric difference.
        \end{enumerate}
    \end{definition}

    \begin{theorem}
        Every group $G$ that satisfies the Følner criterion is sofic.
    \end{theorem}
    \begin{proof}
        Let $F_1, F_2, F_3, \dots $ be a Følner sequence of $G$.
%Notice that $G = \bigcup_{i \in \N} F_i$ by the first property. So $G$ is countable. 
Choose for every $i \in \N: d_i = |F_i|$. We will implicitly identify $\sym(F_i)$ with $\sym(d_i)$. Now choose maps $\Phi_i: F_i \to \sym(F_i)$ such that $\Phi_i(g)(h) = gh$ whenever $gh \in F_i$, which is possible due to \cref{lem:finite_bijections}. 

We first prove the second condition.
Fix $g, h \in G$. By the Følner criterion we know that 
\begin{align*}
    \lim_{i\to \infty} \frac{\card{g^{-1}F_i \triangle F_i}}{\card{F_i}} &= 0\\
    &= \lim_{i\to \infty} \frac{\card{g^{-1}F_i \setminus F_i} + \card{F_i \setminus g^{-1}F_i}}{d_i}\\
    &= \lim_{i\to \infty} \frac{2\card{ F_i \setminus g^{-1}F_i }}{d_i}\\
    &= \lim_{i\to \infty} \frac{2\card{\left\{ x \in F_i \mid gx \notin F_i \right\}}}{d_i}. 
\end{align*}
This implies that \begin{equation*}
    \lim_{i\to \infty} \frac{\card{\left\{ x \in F_i \mid gx \in F_i \right\}}}{d_i} = 1,
\end{equation*}
or equivalently 
\begin{equation*}%\label{eq:proof_folner1}
    \lim_{i\to \infty} \frac{\card{\left\{ x \in F_i \mid gx \notin F_i \right\}}}{d_i} = 0
\end{equation*}
As $\Phi_i(g)(x) = gx$ whenever $gx \in F_i$, we can see as well that 
$$\lim_{i\to \infty} \frac{\left|\left\{ x \in F_i \mid \Phi(g)(x)\ne gx \right\}\right|}{d_i} = 0.$$
Which proves the second condition for soficity. 

To prove the first condition, we can apply \cref{lem:folner_finite_subset}, which gives that \[
	\lim_{i \to \infty} \frac{\card{\{g,h, gh\} F_i \triangle F_i}}{d_i} = 0
.\]
Which means that 
$$\lim_{i\to \infty} \frac{\left|\left\{ x \in F_i \mid gx,hx,ghx \in F_i \right\}\right|}{d_i}= 1.$$
As $\Phi_i(g)(x) = gx$ whenever $gx \in F_i$, we can see as well that
$$\lim_{i\to \infty} \frac{\left|\left\{ x \in F_i \mid (\Phi(g) \circ\Phi(h))(x) = \Phi(gh)(x) \in F_i \right\}\right|}{d_i}= 1.$$
This proves the first condition for soficity and concludes the proof. 

    \end{proof}

    \begin{lemma}\label{lem:finite_bijections} 
        Let $A$ and $B$ be finite sets such that $|A| = |B|$. Given $A' \subset A, B' \subset B$ and a bijection $f': A' \to B'$ there exists a bijection $f: A \to B$ such that $f|_{A'} = f'$. 
    \end{lemma}
    \begin{proof}
        Since $\card A = \card B$ and $\card A' = \card B'$,
        \[
        \card{A \setminus A'} = \card A - \card A' = \card B - \card B' = \card{B \setminus B'}.
        \]
        Therefore there exists a bijection $g: A \setminus A' \to B \setminus B'$. Define $f: A  \to B$ by $f(x) = f'(x)$ for all $x \in A'$ and $f(x)=g(x)$ for all $x \in A \setminus A'$. Then $f$ is a bijection and $f|_{A'} = f'$.
    \end{proof}

    \begin{definition}\cite{noauthor_residually_2018} \label{def:res_fin}
        We say that a group $G$ is residually finite if it adheres to the following equivalent properties.
        \begin{enumerate}
            \item For every $g \in G, g\ne e$ there is an homophorphism $f:G \to H$ such that $f(g) \ne e$.
            \item For every $g \in G, g \ne e$ there is a normal subgroup $H$ such that $g \notin H$.
            \item The intersection of all subgroups of finite index is $\{e\}$.
            \item The intersection of all normal subgroups is $\{e\}$.
            \item $G$ can be embedded in a direct product of finite groups.
        \end{enumerate}
    \end{definition}

    \begin{theorem}
        Every countable residually finite group is sofic.
    \end{theorem}
    \begin{proof}
        Let $G$ be a countable residually finite group. As $G$ is countable we can order the elements non-identity elements $g_1, g_2, g_3, \dots$ .
By the first property of residually finiteness, for every $i=1,2,3, \dots$ there exist a group $G_i$ and a homomorphism $f_i: G \to G_i$ such that $f_i(g_i) \ne 1_G$. Consider the direct products $H_n = \prod_{i = 1}^n G_i$ with the induced homomorphisms $h_i: G \to H_i: g \mapsto (f_1(g), f_2(g), \dots, f_i(g))$. Now we will look at the image of $G$ under these morphisms $\Gamma_i = h_i(G)$.


Notice that $\Gamma_i$ is always finite. We will implicitly identify $\sym(\Gamma_i)$ with $\sym(\card{\Gamma_i})$.
For every $i \in \N$, define $d_i = \card{\Gamma_i}$ and $\Phi_i: G \to \sym(\Gamma_i): g \mapsto (a \mapsto ga)$. We claim that this is a sofic approximation sequence for $G$.

For every $k \in \Gamma_i, g, h \in G$ we can see that $(\Phi_i(g) \circ \Phi_i(h))(k) = ghk = \Phi_i(gh)$. So
\[
\{k \in \Gamma_i \mid (\Phi_i(g) \circ \Phi_i(h))(k) = \Phi_i(gh)\} = \Gamma_i.
\]
Then we can see that for all $g,h \in G$:
$$\lim_{i\to \infty} \frac{1}{d_i} \left|\left\{k \in \Gamma_i| (\Phi_i(g) \circ \Phi_i(h))(k) = \Phi_i(gh)\right\}\right| = 1.$$

For every $g \in G, g\ne 1$ there is an $N \in \N$ such that for every $i > N$, $h_i(G) \ne 1_G$. So it is clear that 
$$\lim_{i \to \infty} \frac{1}{d_i} \card{\left\{k \in \Gamma_i \mid \Phi_i(g) = gk \ne k\right\}} = 1.$$
For $g = 1_G$ we find that 
$$\lim_{i \to \infty} \frac{1}{d_i} \card{\left\{k \in \Gamma_i \mid \Phi_i(g) = k \ne k\right\}} = 0,$$
which proves the second condition. 

    \end{proof}

    \subsection{Examples of Sophic Groups}
    \begin{description}
        \item[Finite groups] Every finite group, $G$ satisfies the Følner criterion, as the taking $(G)_n$ as sequence satisfies the properties in \cref{def:folner}.
        \item[Cyclic groups] All finite cyclic groups are taken care of by the previous example. We only need to check $\Z,+$. This group is residually finite as for every $n \in \Z, n \ne 0$ we can consider the natural morphism $\Z \to \Z /(|n|+1)\Z: x \mapsto x \mod |n|+1$. 
        \item[The finite direct product of Sophic groups] Let $G, H$ be sophic groups with sequence $d_i, e_i$ and maps $\sigma_i, \phi_i$ such that the properties of \cref{def:Sophic} hold. It can be (easily) verified that the sequence $(d_i + e_i)_i$ and maps $\psi_i: G \times H \to \sym(d_i + e_i): x (x \le d_i) \mapsto \sigma_i(x), x (x > d_i) \mapsto \phi_i(x -d_i) + d_i$ satisfy \cref{def:Sophic}.  
    \end{description}
    \paragraph{conjectures}
    \begin{itemize}
        \item subgroups of sophic groups
        \item free product of sophic groups
        \item direct product/sum of infinite groups
    \end{itemize}
    \printbibliography

\end{document}
