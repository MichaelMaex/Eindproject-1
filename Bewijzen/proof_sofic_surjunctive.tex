Suppose that we have a map $\phi: A^G \to A^G$ which is continuous, injective and which commutes with the shifts $\sigma_g$. Now consider $X := \phi(A^G) \subset A^G$. We claim that this is a subshift. Indeed, because $\phi$ is continuous and $A^G$ is compact, $X$ is closed. Furthermore, $X$ is invariant under any shift $\sigma_g$: 
		\begin{align*}
		\sigma_g(X)
		&= (\sigma_g \circ \phi) (A^G) \\
		&= (\phi \circ \sigma_g)(A^G) \\
		&= \phi(A^G).
		\end{align*}
		
The inverse $\phi^{-1}: X \to A^G$ is well defined, continuous (since $A^G$ and therefore also $X$ is Hausdorff) and commutes with the shifts. Therefore,  $\phi^{-1}$ is determined by a finite restriction $\phi^{-1}_0: X_{\Gamma} \to A^{\Gamma}$ with $\Gamma$ a finite subset of $G$. Indeed, since $\phi^{-1}$ is continuous, the value of any component, say $(\phi^{-1}((x_g)_{g \in G}))_e \in A$ is determined by the values $(x_g)_{g \in \Gamma}$ of a finite subset $\Gamma \subset G$, meaning 
		\[
		(\phi^{-1}((x_g)_{g \in G}))_e = (\phi^{-1}_0((x_g)_{g \in \Gamma}))_e
		\]

Now, take a sequence $(x_g)_{g \in G} \in X$ and its restriction $(x_g)_{g \in \Gamma} \in X_{\Gamma}$. Now for any $h \in G$ it is the case that 
		\[
		(\phi^{-1}((x_g)_{g \in G}))_h = (\sigma_h \circ \phi^{-1}((x_g)_{g \in G}))_e  = (\phi^{-1} \circ \sigma_h((x_g)_{g \in G}))_e
		\]

This knowledge allows us to write
		\begin{align*}
		(\phi^{-1}((x_g)_{g \in G}))_h
		&= (\sigma_h \circ \phi^{-1}((x_g)_{g \in G}))_e \\
		&= (\phi^{-1}((x_{gh})_{g \in G}))_e \\
		&= (\phi^{-1}_0((x_{gh})_{g \in \Gamma}))_e \\
		&= (\phi^{-1}_0((x_g)_{g \in \Gamma h}))_e
		\end{align*}
\\
Now fix any $\delta > 0$. Consider $H_n(\delta)$ for any $n \in \N$. Look at any sequence $(a_i)_i \in A^n$. Take then $(x_i)_i = \phi((a_i)_i) \in X$. As $\Gamma$ is only finite it is possible to take $n$ large enough such that $d_{S_n}((a_{\Phi^{-1}_n(g)(i)})_i,(x_{g g_i})_i)) < \delta$ for every $g \in \Gamma$. This is equivalent to saying that for $n \in \N$ large enough $(a_i)_i \in H_n(\delta)$ for every $(a_i)_i \in A^n$ . Therefore, $\limsup_{n \to \infty} |H_n(\delta)| = |A|^{n}$ for any $\delta >0$ and $h_\Sigma(X) = \log |A|$, which, by the previous lemma, means that $X = A^G$. Thus, $\phi$ is surjective.